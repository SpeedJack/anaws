\section{Using Docker}

We have prepared two docker\footnote{\url{https://www.docker.com}} containers,
one for Floodlight and the other for Mininet and Open vSwitch, and a Docker
Compose\footnote{\url{https://docs.docker.com/compose}} configuration. All the
relevant files are under the \code{docker/} directory.

The following versions (or newer) are required:
\begin{itemize}
	\item Docker v17.09.0
	\item Docker Compose v1.17.0
\end{itemize}

From the \code{docker/} folder, Mininet and Floodlight containers can be started
with\footnote{All these commands require root privileges or that the user is in
the \code{docker} group.}:
\begin{verbatim}
$ docker-compose up -d
\end{verbatim}

The first time it is started the Floodlight's container will download all needed
dependencies from the Maven's repository: it may require some time.

To shutdown both containers:
\begin{verbatim}
$ docker-compose down
\end{verbatim}

To view Floodlight's output:
\begin{verbatim}
$ docker logs -f fl
\end{verbatim}

To get a shell inside the Mininet's container:
\begin{verbatim}
$ docker exec -it mn bash
\end{verbatim}
Once inside the Mininet's container, you can find the \code{scripts/} folder
mounted as a volume inside the root's home directory (\code{/root/scripts/}).
Bash scripts to start up Mininet can be ran from here (or just invoke \code{mn}
directly).

To recompile Floodlight, just delete \code{target/floodlight.jar}. Floodlight
will be built again when the container is restarted.
