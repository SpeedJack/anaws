\chapter{Implementation}\label{ch:implementation}

We have implemented the dual ascent routing module for Floodlight latest master
version\footnote{Note that master version of Floodlight requires JDK 8.}
available on Github\footnote{\url{https://github.com/floodlight/floodlight}}.
More precisely, we started with the version at
commit~\href{https://github.com/floodlight/floodlight/tree/d737cb05656a6038f4e2277ffb4503d45b7b29cb}{\code{d737cb0}}.

To know how to get the code and execute it, check
\appendixref{appendix:tutorial}.

The module is implemented in two classes. Both classes are placed in the
\code{net.floodlightcontroller.topology} package:
\begin{itemize}
	\item \code{DualAscentTopologyManager}
	\item \code{DualAscentTopologyInstance}
\end{itemize}

The \code{DualAscentTopologyManager} class is just the adaptation of the
\code{TopologyManager} class for our module. It does not contains any additional
code other than a substitution of all references to \code{TopologyInstance} into
\code{DualAscentTopologyInstance}.

The \code{DualAscentTopologyInstance} class contains code taken from the
\code{TopologyInstance} class except for the \code{dijkstra} method, which has
been substituted with the \code{dualAscent} method that is the core of our
implementation.

The \code{dualAscent} method first performs some initialization of needed data
structure (we have also added the \code{Graph} class used to represent a graph
composed by nodes and directed links). An important part of the initialization
process is choosing the topology root of the dual ascent algorithm, as shown in
\lstref{lst:isdstrooted}.

\lstinputlisting[language=java, style=javacode, label={lst:isdstrooted},
caption={Initialization of the root of the topology}, firstline=14,
lastline=14]{dualAscent.java}

Then, the most relevant part that implements the dual ascent algorithm is shown
in \lstref{lst:dualascentmain}.

\lstinputlisting[language=java, style=javacode, label={lst:dualascentmain},
caption={Main cycle of the dual ascent algorithm}, firstline=31,
lastline=38]{dualAscent.java}

The \code{findRootComp} method searches for a root component in the auxiliary
graph. If no root component is found, it returns \code{null}.

The \code{findMinArc} method returns the arc with minimum cost that is in the
complete graph of the topology but not in the auxiliary graph. If it returns
\code{null}, meaning that there is no link satisfying the requirements, the
current root component is added to a set in order to instruct the
\code{findRootComp} method to not return the same root component again avoiding
an infinite cycle.

The \code{editCosts} method updates the table of reduced costs which is used by
the \code{findMinArc} method whenever a minimum cost arc has to be retrieved
from the complete graph of the topology.

Finally, the \code{addArc} method adds \code{minLink} to the list of links in
the auxiliary graph and updates the \code{connections} matrix that maintains the
list of nodes that are connected (directly or indirectly) in the auxiliary
graph. This matrix is inspected by the \code{areConnected} method which is
mainly used by the \code{findRootComp} to identify root components.

When there are no new root components, the loop ends. Then, all nodes that are
not connected (directly or indirectly) with the root node are removed from the
auxiliary graph, as shown in \lstref{lst:dualascentnodescleanup}, since they are
not needed.

\lstinputlisting[language=java, style=javacode,
label={lst:dualascentnodescleanup}, caption={Nodes disconnected from the root
node are removed from the auxiliary graph}, firstline=40,
lastline=45]{dualAscent.java}

After that, the code shown in \lstref{lst:dualascentinvertlinks} is used to
consider the case where the root is the destination of the method, because in
this case the root is required to be the node that all other nodes converge to,
and since the original dual ascent computes a path from the root to all other
target nodes, in this case all paths must start from target nodes and end to the
root node. This operation is possible thanks to the fact that in the complete
graph there are always links in the opposite direction.

\lstinputlisting[language=java, style=javacode,
label={lst:dualascentinvertlinks}, caption={If the root node is the destination
requested by Floodlight, the links are inverted}, firstline=47,
lastline=67]{dualAscent.java}

Finally the final broadcast tree is built. This is done using the method
\code{buildBroadcastTree} which resolves the minimum spanning tree problem using
the Dijkstra's algorithm (this is necessary to remove cycles in the final
broadcast tree).
