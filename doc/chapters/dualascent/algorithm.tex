\section{The algorithm}\label{sec:algorithm}

The dual ascent algorithm starts from the root node and works
iteratively. There are three important data structures:

\begin{description}
	\item[Root Component] A set of strongly connected nodes, containing at
		least a target node, which hasn't incoming links from other
		target nodes or from the root node.
	\item[Auxiliary Graph] A graph which starts with all the topology nodes
		(including the non-target ones) and no links. At each iteration
		of the algorithm, the link with the minimum cost is added.
	\item[Reduced Cost Table] A table which stores the costs of connected
		nodes. Element \(R_i^j\) represents the dual cost to reach the
		node \(i\), selecting as root component the node \(j\). The
		final cost of the algorithm is the sum of all \(R_k^k\) values
		(slacks), where \(k\) is the index of a target node.
\end{description}

At each step a list of current root components is extracted from the auxiliary
graph. For each element of the list the algorithm finds, from the starting
topology, the link with minimum cost that has the root component as destination
and adds it to the auxiliary graph. Then the reduced cost table is updated by
decreasing the new cost value with the one that involved that node, selected at
the first iteration of the algorithm (the minimum one). The value obtained is
the \emph{slack}, that is added into \(R_j^j\) elements. Also values \(R_i^j\)
are updated considering the new paths formed in the auxiliary graph.

These steps are done until there are no remaining root components or no new
links from the starting topology. If in the auxiliary graph there are any
cycles, they are resolved deleting highest cost paths, so that the final result
is a broadcast tree including all the target nodes with zero or some non-target
ones.
