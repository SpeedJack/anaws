\chapter{Tutorial}\label{appendix:tutorial}

The following works with any Linux distribution with proper kernel modules
installed (usually already present in major distributions).

To develop the module, we have forked the Floodlight's repository. The fork is
available on Github at \url{https://github.com/SpeedJack/anaws}. The module is
available on the \code{dualascent} branch (the default branch).

\begin{verbatim}
$ git clone --recurse-submodules \
    https://github.com/SpeedJack/anaws.git
\end{verbatim}

Prerequisites for Floodlight:
\begin{itemize}
	\item JDK 8 (it does not work on more recent versions).
	\item Python with its header files (\code{python-dev}).
	\item Maven (unfortunately, build with \code{ant} is broken due to
		upstream not updating their dependencies' snapshots).
\end{itemize}

We have set up a docker environment to build and run Floodlight and Mininet (see
below). Using the docker containers is recommended, since this is the setup we
have used. Anyway, if docker is not an option, Floodlight can be built with:

\begin{verbatim}
$ mvn package -Dmaven.test.skip=true
\end{verbatim}

To run Floodlight:

\begin{verbatim}
$ ./floodlight.sh
\end{verbatim}

Regarding Mininet and Open vSwitch, we have used the following versions:
\begin{itemize}
	\item Mininet v2.2.2
	\item Open vSwitch 2.13.1
\end{itemize}

Inside the \code{scripts/} folder there are various Bash scripts useful to run
Mininet with various topologies. Other scripts that deserves a mention:
\begin{itemize}
	\item \code{docker/switch-algorithm.sh} that can be used to switch
		between the Dijkstra and the Dual Ascent algorithms in the
		Floodlight's configuration.
	\item \code{docker/set-*.sh} useful to change other configuration
		options of Floodlight by passing the new value as argument.
	\item \code{docker/allow-xterms.sh} used to allow the execution of the
		Mininet's \code{xterm} command from inside the container.
	\item \code{scripts/collectbw.sh} is the script used to measure the
		host-to-host bandwidth. It is meant to be run from inside
		mininet hosts.
\end{itemize}

\section{Using Docker}

We have prepared two docker\footnote{\url{https://www.docker.com}} containers,
one for Floodlight and the other for Mininet and Open vSwitch, and a Docker
Compose\footnote{\url{https://docs.docker.com/compose}} configuration. All the
relevant files are under the \code{docker/} directory.

The following versions (or newer) are required:
\begin{itemize}
	\item Docker v17.09.0
	\item Docker Compose v1.17.0
\end{itemize}

From the \code{docker/} folder, Mininet and Floodlight containers can be started
with\footnote{All these commands require root privileges or that the user is in
the \code{docker} group.}:
\begin{verbatim}
$ docker-compose up -d
\end{verbatim}

Docker Compose is just an utility to ease the setup of containers, it is not
strictly necessary to run the containers. Although not recommended, if Compose
is not available, containers can be started using only Docker. The following
\emph{should} be ok (starting from the \code{docker/} directory):
\begin{verbatim}
$ cd mn && docker build -t mininet .
$ cd ../fl && docker build -t floodlight .
$ cd ../../
$ docker network create docker_flmnnet
$ docker volume create docker_fl-m2
$ docker run -d -p "8080:8080" -p "1044:1044" \
    --rm -v "$(pwd)/:/usr/src/myapp/" \
    -v "docker_fl-m2:/root/.m2" \
    --network docker_flmnnet --name fl floodlight
$ docker run -d --rm -v "$(pwd)/scripts/:/root/scripts/" \
    -v "/lib/modules:/lib/modules" \
    -v "/tmp/.X11-unix:/tmp/.X11-unix" -e DISPLAY \
    --tty --privileged --link fl --network docker_flmnnet \
    --name mn mininet
\end{verbatim}

Note that the last two commands make use of the \code{pwd} command. It assumes
that the user is in the project's root directory.

The first time it is started the Floodlight's container will download all needed
dependencies from the Maven's repository: it may require some time.

To shutdown both containers:
\begin{verbatim}
$ docker-compose down
\end{verbatim}

Or, without Docker Compose:
\begin{verbatim}
docker stop mn
docker stop fl
\end{verbatim}

To view Floodlight's output:
\begin{verbatim}
$ docker logs -f fl
\end{verbatim}

To get a shell inside the Mininet's container:
\begin{verbatim}
$ docker exec -it mn bash
\end{verbatim}
Once inside the Mininet's container, you can find the \code{scripts/} folder
mounted as a volume inside the root's home directory (\code{/root/scripts/}).
Bash scripts to start up Mininet can be ran from here (or just invoke \code{mn}
directly).

To recompile Floodlight, just delete \code{target/floodlight.jar}. Floodlight
will be built again when the container is restarted.

When the Floodlight's container is up, the Floodlight's web UI can be accessed
at:

\begin{center}
	\url{http://127.0.0.1:8080/ui/pages/}
\end{center}

To capture traffic with Wireshark, first enter in the Mininet container and set
the root's password to whatever:
\begin{verbatim}
$ docker exec -it mn bash
# passwd
\end{verbatim}

Then, configure Wireshark to capture through the \code{sshdump} utility. If the
``mn'' hostname does not resolve to the IP address of the container, the address
can be viewed with:
\begin{verbatim}
$ docker inspect mn | grep IPAddress
\end{verbatim}

